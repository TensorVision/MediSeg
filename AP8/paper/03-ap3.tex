%!TEX root = vorlage.tex

\subsection{Base line experiments}\label{sec:ap3}

TODO: Constant background? Accuracy, Precision, Recall?

The most basic information for pixel-wise semantic segmentation is the color of
the pixel. Typically, images are in RGB format. This means the image has
three~channels ({\color{red} R}ed, {\color{green} G}reen, {\color{blue} B}lue).
Each channel has 8~bit and thus $2^8 = 256$ possible values, ranging from 0 to
255. This gives ${(2^8)}^3 = \num{16777216}$ possible colors. Obviously, only
the color can not give a perfect result in all circumstances as the measured
color changes due to smoke, shaddows, specular~highlights and insufficient
illumination. But it gives an impression how important local features are for
the specific problem.

A model with 64~sigmoid nodes in a first hidden layer with $\SI{50}{\percent}$
dropout, 64~sigmoid nodes in a second hidden layer with $\SI{50}{\percent}$ and
one sigmoid output unit achieved a pixel-wise accuracy
of~$\SI{92.88}{\percent}$\footnote{This is the same as the DICE coefficient.},
a precision of $\SI{76.13}{\percent}$ and a recall of $\SI{32.94}{\percent}$.
The confusion matrix is given in~\cref{table:cm-model-301}.
