%!TEX root = vorlage.tex

\subsection{Base line experiments}\label{sec:ap3}

The most basic information for pixel-wise semantic segmentation is the color of
the pixel. Typically, images are in RGB format. This means the image has
three~channels ({\color{red} R}ed, {\color{green} G}reen, {\color{blue} B}lue).
Each channel has 8~bit and thus $2^8 = 256$ possible values, ranging from 0 to
255. This gives ${(2^8)}^3 = \num{16777216}$ possible colors for each pixel.
Obviously, only the color can not give a perfect result in all circumstances as
the measured color changes due to smoke, shadows, specular~highlights and
insufficient illumination. But it gives an impression how important local
features are for the specific problem.

A model with 64~sigmoid nodes in a first hidden layer with $\SI{50}{\percent}$
dropout~\cite{srivastava2014dropout}, 64~ReLu nodes in a second hidden layer
with $\SI{50}{\percent}$ dropout and one sigmoid output unit was used as a
baseline.

The architecture of the baseline model is visualized
in~\cref{fig:baseline-architecture}. Neither preprocessing nor data
augmentation were applied.

The baseline model achieved a pixel-wise accuracy
of~$\SI{92.88}{\percent}$,\footnote{This is the same as the DICE coefficient.}
a precision of $\SI{76.13}{\percent}$ and a recall of $\SI{32.94}{\percent}$.
The confusion matrix is given in~\cref{table:cm-model-301}.

\begin{figure}[ht]
    \centering
    \input{mlp-architecture}
    \caption{Architecture of the baseline model.}
    \label{fig:baseline-architecture}
\end{figure}
