%!TEX root = vorlage.tex

\section{Introduction}\label{sec:introduction}
Operations like the removal of a tumor or the gallbladder (a cholecystectomy)
require the body of the patient to be opened. However, minimal-invasive
operations got more and more attentions since~1987~\cite{wickham1987new}. In
this kind of operation, the surgeon tries to make as little and as small cuts
in the patients body as possible. The advantage is that the patients skin can
heal faster and thus the patient can recover faster from the damage which was
done by the operation. The disadvantage of minimal-invasive operations is that
the operation itself gets harder for the surgeon. Special medical equipment has
to be used: Small cameras and fiber optic cables so that the surgeons can see
what their doing, % TODO: http://health.stackexchange.com/q/7304/2445

Machines can not only provide the possibility to make more fine-grained
movements (e.g. with the \textit{da Vinci} Surgical System (Intuitive Surgical,
Mountain View, Calif)), but also improve vision. For example, the limited
visibility due to cautarization-induced smoke can be fought by highlighting the
medical instruments, the instruments themselves can be recognized and the
operation phase can be detected. If the camera images had a pixel-wise
segmentation of medical instruments and background, those tasks would be
simpler.
